\documentclass[14pt,a4paper]{extarticle}
\usepackage[utf8]{inputenc}

\usepackage[english,russian]{babel}
\usepackage{indentfirst}
\usepackage{amsmath}
\usepackage{amsfonts}
\usepackage{amssymb}
\usepackage{color}
\usepackage{float}
\usepackage{placeins}
\usepackage{mathrsfs}



\usepackage[version=3]{mhchem}% Package for chemical equation typesetting
%\renewcommand{\thefigure}{\thechapter.\Alph{figure}}


\usepackage{lineno}
%\linenumbers

\usepackage[usenames,dvipsnames,svgnames,table]{xcolor}
\usepackage[hypcap]{subcaption}
% \usepackage[colorlinks=true, linkcolor=Maroon, citecolor=ForestGreen]{hyperref}
\usepackage[colorlinks=false]{hyperref}
\usepackage{graphicx}
\usepackage{subcaption}
\usepackage[height=25cm, a4paper, hmargin={2.5cm, 1.5cm}, vmargin={2cm, 2.3cm}]{geometry}
\setlength{\parindent}{1em}
\setlength{\parskip}{0pt}% {.3em}

\usepackage{titlesec}
\titleformat*{\section}{\normalsize\bfseries}
\titleformat*{\subsection}{\normalsize\bfseries}
%\titlelabel{\thetitle.\quad}




\usepackage[separate-uncertainty = true, multi-part-units=single, range-phrase=--, range-units=single]{siunitx}

\mathchardef\period=\mathcode`.
\DeclareMathSymbol{.}{\mathord}{letters}{"3B}

\usepackage{setspace}
\onehalfspacing

\usepackage[backend=bibtex, style=numeric, sorting=none, maxnames=3, minnames=3, hyperref=true, url=true, isbn=false, backref=true]{biblatex}
\bibliography{mybib}

\usepackage{tocloft}
\usepackage{tikz}
\hyphenation{Ha-ma-ma-tsu}

\newcommand{\RNum}[1]{\uppercase\expandafter{\romannumeral #1\relax}}

\usepackage{titletoc}
\dottedcontents{section}[3em]{}{2.9em}{1pc}
\dottedcontents{subsection}[3em]{}{2.3em}{1pc}
\renewcommand{\thesection}{\thechapter\arabic{section}.}
%\renewcommand{\thesection}{\thechapter\arabic{subsection}.}

\renewcommand{\figurename}{Рис.}
\usepackage[labelsep=period, font=small]{caption}

\usepackage{titlesec}
\titlespacing*{\section}
{0pt}{1pt}{0pt}
% {0pt}{5.5ex plus 1ex minus .2ex}{4.3ex plus .2ex}
\titlespacing*{\subsection}
{0pt}{1pt}{0pt}
% {0pt}{5.5ex plus 1ex minus .2ex}{4.3ex plus .2ex}

\usepackage{fancyhdr}

\pagestyle{fancy}
\fancyhf{}
%\fancyhead[LE,RO]{Share\LaTeX}
%\fancyhead[RE,LO]{Guides and tutorials}
%\fancyfoot[CE,CO]{\leftmark}
\fancyfoot[LE,RO]{\thepage}
\renewcommand{\headrulewidth}{0pt}


\begin{document}

Выведем спектр излучения из ондулятора. Начнём с уравнения движение релятивистского электрона в магнитном поле.

\begin{equation}
	\vec{F} = e \cdot[\vec{v} \times \vec{B}],
\end{equation} 
где $e$ --- заряд электрона, а $\vec{v}$ и $\vec{B}$ скорость частицы и магнитное поле соответственно. Уравнение можно переписать в виде:

\begin{equation}
	\frac{d\vec{p}}{dt} = \frac{e}{\gamma m_e}\cdot [\vec{v} \times \vec{B}],
\end{equation}
где $\gamma$ --- лоренц фактор, появившийся из релятивистского импульса. Направим ось $z$ вдоль направления релятивистского движения электрона и введём магнитное поле в ондуляторе $B_0\cos(k_w z)$, направленное вдоль оси $y$, где $k_w$ связана с периодом ондулятора следующим образом $k_w = \frac{2\pi}{\lambda_w}$. 

\begin{equation}
	\begin{cases}
		\frac{d^2 x}{dt^2} = - \frac{e B_0}{\gamma m_e}\frac{dz}{dt} \cos(k_w z)\\
		\frac{d^2 z}{dt^2} = \frac{e B_0}{\gamma m_e}\frac{dx}{dt} \cos(k_w z)
	\end{cases} 
\end{equation}

один раз интегрируя первой уравнение из системы с заменой $dz = \beta cdt$, где $\beta = \frac{\|\vec{v}\|}{c}$, можно получить: 

\begin{equation}
 	\label{eq:dx/dt}
	\frac{dx}{dt} = - \frac{eB_0}{\gamma m_ek_w} \sin(k_w z)
\end{equation}
Введём известный коэффициент ондуляторности --- $K = \frac{eB_0 \lambda}{2\pi m_ek_w}$, который показывает угол отклонения электрона от оси $z$. 

Подставляя получившийся результат~\ref{eq:dx/dt} во второе уравнение системы и интегрируя с пределами интегрирования от $0$ до некоторого $z_0$, получим систему:

\begin{equation}
	\begin{cases}
		\frac{dx}{dt} = - \frac{Kc}{\gamma} \sin(k_w z)\\
		\frac{dz}{dt} = \beta c - \frac{K^2 c}{2 \gamma^2 \beta}\sin^2(k_w z)
	\end{cases} 
\end{equation}

\end{document}
























